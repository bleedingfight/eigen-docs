\section{map对象}
\begin{cpp}
Map<Matrix<typename Scalar, int RowsAtCompileTime, int ColsAtCompileTime> >
\end{cpp}
构造 Map 变量,你需要两个信息:
\begin{itemize}
\item 指向定义数组、向量数据的指针。
\item 数组、向量的形状。
\end{itemize}
你可以这样定义:
\begin{cpp}
Map<MatrixXf> mf(pf,rows,columns);
\end{cpp}
这里的pf是指向浮点数的数组数据的指针。你也可以使用固定大小的整数向量:
\begin{cpp}
Map<const Vector4i>mi(pi)
\end{cpp}
这里的pi是指向int的指针。这种情况不用指定size,因为size已经在Matrix、Array类型的对象中指定的。
\begin{remark}
Map 没有默认的构造函数,你必须传一个指针初始化对象。
\end{remark}
Map 可以容纳多种数据表示,有两个可选的模板选项。
\begin{cpp}
Map<typename MatrixType,int MapOptions,typename StrideType>
\end{cpp}
\begin{itemize}
\item \textbf{MapOptions} 指定是否指针是 \href{https://eigen.tuxfamily.org/dox/group__enums.html#gga45fe06e29902b7a2773de05ba27b47a1ae12d0f8f869c40c76128260af2242bc8}{Aligned}或者是\href{https://eigen.tuxfamily.org/dox/group__enums.html#gga45fe06e29902b7a2773de05ba27b47a1a4e19dd09d5ff42295ba1d72d12a46686}{Unaligned},默认是Unaligned。
\item StrideType允许你指定自定义的内存布局,使用\href{https://eigen.tuxfamily.org/dox/classEigen_1_1Stride.html}{Stride}类。
\end{itemize}
\cppfile[firstline=9,lastline=12]{samples/test/src/test_map.cc}
输出如下:
\begin{bash}
列优先:
0 2 4 6
1 3 5 7
行优先
0 1 2 3
4 5 6 7
使用stride之后的的行优先:
0 1 2 3
4 5 6 7
\end{bash}
你可以想使用其他Eigen对象一样使用Map对象:
\cppfile[firstline=13,lastline=33]{samples/test/src/test_map.cc}
所有的Eigen函数都能接收Map对象。你自己写的函数不能自动接收Eigen对象(\href{接收Eigen参数}{https://eigen.tuxfamily.org/dox/TopicFunctionTakingEigenTypes.html})。你可以在声明之后改变Array对象。

