\section{Matrix}
Matrix 是一个模板函数,其中包含6个参数。
\begin{cpp}
template<typename _Scalar, int _Rows, int _Cols, int _Options, int _MaxRows, int _MaxCols>
class Eigen::Matrix< _Scalar, _Rows, _Cols, _Options, _MaxRows, _MaxCols >
\end{cpp}
\begin{itemize}
\item \_Scalar:数值类型,例如:float、 double、 int 或者是 std::complex<float>.也支持用户自定义。
\item \_Rows	行数或者为 Dynamic
\item \_Cols	列数或者为 or Dynamic
\item \_Options:数据存放方式,既可以自动对齐也可以不对齐,默认是按列存储。默认是对齐的除非fixed的大小不是packet的整数倍。
\item \_MaxRows:最大行数,默认为 \_Rows。
\item \_MaxCols:最大的列数,默认为 \_Cols。 
\end{itemize}
自带的数据类型有如下这么多种,在size大约为16时静态存储的性能很好,size大于32的时候使用静态存储性能收益将变得较低。
\begin{itemize}
\item \cppinline{using 	Eigen::Matrix2 = Matrix< Type, 2, 2 >}
\item \cppinline{using 	Eigen::Matrix2X = Matrix< Type, 2, Dynamic >}
\item \cppinline{using 	Eigen::Matrix3 = Matrix< Type, 3, 3 >}
\item \cppinline{using 	Eigen::Matrix3X = Matrix< Type, 3, Dynamic >}
\item \cppinline{using 	Eigen::Matrix4 = Matrix< Type, 4, 4 >}
\item \cppinline{using 	Eigen::Matrix4X = Matrix< Type, 4, Dynamic >}
\item \cppinline{using 	Eigen::MatrixX = Matrix< Type, Dynamic, Dynamic >}
\item \cppinline{using 	Eigen::MatrixX2 = Matrix< Type, Dynamic, 2 >}
\item \cppinline{using 	Eigen::MatrixX3 = Matrix< Type, Dynamic, 3 >}
\item \cppinline{using 	Eigen::MatrixX4 = Matrix< Type, Dynamic, 4 >}
\item \cppinline{using 	Eigen::RowVector = Matrix< Type, 1, Size >}
\item \cppinline{using 	Eigen::RowVector2 = Matrix< Type, 1, 2 >}
\item \cppinline{using 	Eigen::RowVector3 = Matrix< Type, 1, 3 >}
\item \cppinline{using 	Eigen::RowVector4 = Matrix< Type, 1, 4 >}
\item \cppinline{using 	Eigen::RowVectorX = Matrix< Type, 1, Dynamic >}
\item \cppinline{using 	Eigen::Vector = Matrix< Type, Size, 1 >}
\item \cppinline{using 	Eigen::Vector2 = Matrix< Type, 2, 1 >}
\item \cppinline{using 	Eigen::Vector3 = Matrix< Type, 3, 1 >}
\item \cppinline{using 	Eigen::Vector4 = Matrix< Type, 4, 1 >}
\item \cppinline{using 	Eigen::VectorX = Matrix< Type, Dynamic, 1 >}
\end{itemize}
\begin{remark}
	\begin{itemize}
		\item N是2,3,4或者X(表示动态存储)
		\item 通常i表示整数,f表示浮点数、d表示双精度浮点数,cf表示complex<float>,cd表示complex<double>。虽然上咩只列出了这么些种类型,但是并不意味着Eigen仅仅支持这么多中,实际上Eigen支持很多自定义的类型。
	\end{itemize}
\end{remark}
通常你可以很轻松的构建一个 Matrix 对象然后使用逗号分隔的初始化列表初始化你的 Matrix 。例如:
\cppfile{code/matrix_demo1.cc}
输出如下:
\begin{bash}
 0.680375   0.59688 -0.329554
-0.211234  0.823295  0.536459
 0.566198 -0.604897 -0.444451
1.2 1.2 1.2
1.2 1.2 1.2
1.2 1.2 1.2
vec1 =
1
2
3
vec2 =
1
2
3
\end{bash}
Matrix 对象实现了常见的操作包括矩阵-矩阵的四则运算、矩阵-标量四则运算、矩阵本身的转置、共厄矩阵、inplace的转置、矩阵和向量的乘法、点乘和叉乘(三维向量)、常见的reduce操作等等。常见的函数如下\tabref{tab:matrix_func}:
\begin{table}[!htbp]
	\centering
	\caption{Matrix 常见的方法}
	\begin{tabular}{ccc}
		\toprule
		方法&作用&说明\\
		\midrule
		m.transpose()&转置&非原位转置\\
		m.adjoint()&矩阵的伴随矩阵&\\
		m.cross()&叉乘&\\
		m.sum()&求和&\\
		m.transposeInPlace()&矩阵的原位转置\\
		m.prod()&所有元素的乘积&\\
		m.conjugate()&矩阵的共厄矩阵&\\
		m.mean()&所有元素的均值&\\
		m.minCoeff()&所有元素中的最小元素&\\
		m.maxCoeff()&所有元素中的最大元素&\\
		m.trace()&矩阵的迹&\\
		m.dot()&按点相乘&\\
		\bottomrule
	\end{tabular}
	\label{tab:matrix_func}
\end{table}
\cppfile{code/matrix_demo2.cc}
